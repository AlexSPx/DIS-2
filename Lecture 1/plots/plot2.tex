\documentclass[../lecture-1.tex]{subfiles}

\usepackage[utf8]{inputenc}
\usepackage[bulgarian]{babel}

\usepackage{amsmath}
\usepackage{amsfonts}
\usepackage{amssymb}
\usepackage{mathtools}

\usepackage{pgfplots}
\begin{document}
\pgfplotsset{
    right segments/.code={\pgfmathsetmacro\rightsegments{#1}},
    right segments=3,
    right/.style args={#1:#2}{
        ybar interval,
        domain=#1+((#2-#1)/\rightsegments):#2+((#2-#1)/\rightsegments),
        samples=\rightsegments+1,
        x filter/.code=\pgfmathparse{\pgfmathresult-((#2-#1)/\rightsegments)}
    }
}

\pgfplotsset{
    upperLimit segments/.code={\pgfmathsetmacro\upperlimitsegments{#1}},
    upperLimit segments=3,
    upperLimit/.style args={#1:#2}{
        ybar interval,
        domain=#1:#2,
        samples=\upperlimitsegments+1,
        x filter/.code=\pgfmathparse{\pgfmathresult}
       }
}

\begin{tikzpicture}[/pgf/declare function={f=-15*(x-5)+(x-5)^3+40;}]
\begin{axis}[
    scale=0.65,
    domain=0:9,
    samples=100,
    axis lines=middle,
    xtick distance=1,
    xticklabels={,,,,a,,,,b},
    yticklabels={},
    ymin=-1,
    ymax=70
]
\addplot [thick] {f};
\addplot [
    red,
    upperLimit segments=6,
    upperLimit=3:7
] {f};
\addplot [
    blue,
    right segments = 6,
    right=3:7
] {f};
\addplot [thick] {f};
\end{axis}
\end{tikzpicture}
\end{document}