\documentclass{article}

\usepackage[utf8]{inputenc}
\usepackage[bulgarian]{babel}

\usepackage{amsmath}
\usepackage{amsfonts}
\usepackage{amssymb}
\usepackage{mathtools}

\usepackage{pgfplots}
\usepackage{xcolor}


\usepackage{subfiles}

\author{hornyta}

\title{DIS 2 Лекция 1}

\begin{document}

\maketitle

\section{Определен интеграл}

    \begin{minipage}{0.39\linewidth}
        \subfile{./plots/plot1.tex}
    \end{minipage}
    \begin{minipage}{0.6\linewidth}
        \(f:[a,b] \to \mathbb{R} \), ограничена \\
        \(\sup\{ f(x) : x \in [a, b] \} (b-a)\) - горна оценка\\
        \(\inf\{ f(x) : x \in [a, b] \} (b-a)\) - долна оценка
        Лицето на фигурата варира между тези две стойности.
    \end{minipage}
    Правим подразбиване на интервала \([a, b] \tau: a=x_0<x_1<x_2<\dots<x_n=b\) 
    (което е крайна редица)
    
    \begin{minipage}{0.39\linewidth}
        \subfile{./plots/plot2.tex}
    \end{minipage}
    \begin{minipage}{0.6\linewidth}
        \begin{center}
            Разглеждаме \([x_{i-1}, x_i]\)
        \end{center}
        \[M_j := \sup\{f(x) : x \in [x_{i-1}, x_i] \}\]
        \[m_j := \inf\{f(x) : x \in [x_{i-1}, x_i] \}\]
    \end{minipage}
    \(S_f(\tau) := \sum_{j=1}^{n} M_j(x_j-x_{j-1})\) - Голяма сума на Дарбу за \(f\) при подр. \(\tau\)
    \(s_f(\tau) := \sum_{j=1}^{n} m_j(x_j-x_{j-1})\) - Малка сума на Дарбу за \(f\) при подр. \(\tau\) 

    \paragraph{Лема 1} Ако \(\tau^* \ge \tau \), то \(S_f(\tau^*) \le S_f(\tau)\) и \(s_f(\tau^*) \le s_f(\tau)\)
    \(\tau^*, \tau\) - 2 подразбивания, където \(\tau^*\) е "по-фино" от \(\tau\) (\(\tau^*\) съдържа всичките елементи на \(\tau\))
    \subparagraph{Доказателство:} б.о.о \(\tau^*\) се получава от \(\tau\) с прибавяне на една точка
    \[\tau: a=x_0<x_1<\dots<x_n=b\]
    \[\tau^*: a=x_o<x_1<\dots<x_{i-1}<x^*<x_i<\dots<x_n=b\]
    
    \begin{equation*}
        \begin{split}
                & S_f(\tau)-S_f(\tau^*) = \\
                & \sum_{j=1}^{n}\sup_{[x_{j-1},x_j]}f \cdot (x_j-x_{j-1}) -
                \sum_{j=1}^{i-1}\sup_{[x_{j-1},x_j]}f \cdot (x_j-x_{j-1}) \\
                & - \sup_{[x_{i-1},x^*]}f \cdot (x^*-x_{i-1}) -
                \sup_{[x^*,x_i]}f \cdot (x_i-x^*) \\
                & - \sum_{j=1}^{n}\sup_{[x_{j-1},x_j]}f \cdot (x_j-x_{j-1}) =\\
                & = 
                \sup_{[x_{i-1},x_i]}f \cdot (x_j-x_{i-1})
                - \sup_{[x_{i-1}, x^*]}f \cdot (x^*-x_{i-1}) 
                - \sup_{[x^*,x_i]}f \cdot (x^*-x_i) \ge \\
                & = 
                \sup_{[x_{i-1},x_i]}f \cdot (x_j-x_{i-1})
                - \sup_{[x_{i-1}, x_i]}f \cdot (x^*-x_{i-1}) 
                - \sup_{[x_{i-1},x_i]}f \cdot (x^*-x_i) = 0 \\
                \square
        \end{split}
    \end{equation*}

\end{document}