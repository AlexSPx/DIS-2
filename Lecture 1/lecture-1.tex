\documentclass{article}

\usepackage[utf8]{inputenc}
\usepackage[bulgarian]{babel}

\usepackage{stix2}
\usepackage{amsmath}
\usepackage{amsfonts}
\usepackage{amssymb}
\usepackage{mathtools}
\usepackage{pgfplots}
\usepackage{xcolor}
\usepackage{subfiles}
\usepackage{nameref}
\usepackage{hyperref}
\usepackage{tcolorbox}
\hypersetup{allbordercolors={white},allcolors={white}}

\newcommand{\tvurdenie}[2]{
    \begin{tcolorbox}[title = #1 ,colframe = blue!70!black, colback = blue!10!white]
        #2
    \end{tcolorbox}
}
\newcommand{\opredelenie}[2]{
    \begin{tcolorbox}[title = #1 ,colframe = red!70!black, colback = red!10!white]
        #2
    \end{tcolorbox}
}
\newcommand{\primer}[2]{
    \begin{tcolorbox}[title = #1 ,colframe = blue!70!black, colback = blue!10!white]
        #2
    \end{tcolorbox}
}

\author{hornyta}

\title{DIS 2 Лекция 1}

\begin{document}

\maketitle

\section{Определен интеграл}

\begin{minipage}{0.39\linewidth}
    \subfile{./plots/plot1.tex}
\end{minipage}
\begin{minipage}{0.6\linewidth}
        \(f:[a,b] \to \mathbb{R} \), ограничена \\
        \(\sup\{ f(x) : x \in [a, b] \} (b-a)\) - горна оценка\\
        \(\inf\{ f(x) : x \in [a, b] \} (b-a)\) - долна оценка
        Лицето на фигурата варира между тези две стойности.
\end{minipage}
Правим подразбиване на интервала \([a, b] \tau: a=x_0<x_1<x_2<\dots<x_n=b\)
(което е крайна редица)
    
\begin{minipage}{0.39\linewidth}
    \subfile{./plots/plot2.tex}
\end{minipage}
\begin{minipage}{0.6\linewidth}
    \begin{center}
        Разглеждаме \([x_{i-1}, x_i]\)
    \end{center}
    \[M_j := \sup\{f(x) : x \in [x_{i-1}, x_i] \}\]
    \[m_j := \inf\{f(x) : x \in [x_{i-1}, x_i] \}\]
\end{minipage}
\opredelenie{Суми на Дарбу}{
\(S_f(\tau) := \sum_{j=1}^{n} M_j(x_j-x_{j-1})\) - Голяма сума на Дарбу за \(f\) при подр. \(\tau\)
\(s_f(\tau) := \sum_{j=1}^{n} m_j(x_j-x_{j-1})\) - Малка сума на Дарбу за \(f\) при подр. \(\tau\)
}
\tvurdenie{Лема 1}{Ако \(\tau^* \subseteq \tau \), то \(S_f(\tau^*) \le S_f(\tau)\) и \(s_f(\tau^*) \le s_f(\tau)\)
    \\\(\tau^*, \tau\) - 2 подразбивания, където \(\tau^*\) е "по-фино" от \(\tau\) (\(\tau^*\) съдържа всичките елементи на \(\tau\))
}
\underline{Доказателство:} (б.о.о) \(\tau^*\) се получава от \(\tau\) с прибавяне на една точка
\[\tau: a=x_0<x_1<\dots<x_n=b\]
\[\tau^*: a=x_o<x_1<\dots<x_{i-1}<x^*<x_i<\dots<x_n=b\]
    
\begin{equation*}
\begin{split}
                & S_f(\tau)-S_f(\tau^*) = \\
                & \sum_{j=1}^{n}\sup_{[x_{j-1},x_j]}f \cdot (x_j-x_{j-1}) -
                \sum_{j=1}^{i-1}\sup_{[x_{j-1},x_j]}f \cdot (x_j-x_{j-1}) \\
                & - \sup_{[x_{i-1},x^*]}f \cdot (x^*-x_{i-1}) -
                \sup_{[x^*,x_i]}f \cdot (x_i-x^*) \\
                & - \sum_{j=1}^{n}\sup_{[x_{j-1},x_j]}f \cdot (x_j-x_{j-1}) =\\
                & = 
                \sup_{[x_{i-1},x_i]}f \cdot (x_j-x_{i-1})
                - \sup_{[x_{i-1}, x^*]}f \cdot (x^*-x_{i-1}) 
                - \sup_{[x^*,x_i]}f \cdot (x^*-x_i) \ge \\
                & = 
                \sup_{[x_{i-1},x_i]}f \cdot (x_j-x_{i-1})
                - \sup_{[x_{i-1}, x_i]}f \cdot (x^*-x_{i-1}) 
                - \sup_{[x_{i-1},x_i]}f \cdot (x^*-x_i) = 0 \\
                \square
\end{split}
\end{equation*}

\tvurdenie{Лема 2}{
    $\tau_1, \tau_2$ произволни подразбивания на $[a,b]$.
    Тогава $s_f(\tau_1)\leq S_f(\tau_2)$.
}
\underline{Доказателство:} Съществува $\tau^*$, такова че $\tau^*\supseteq \tau_1, \tau*\supseteq \tau_2$.
По Лема 1 $s_f(\tau^*)\leq s_f(\tau_1)$ и $S_f(\tau^*)\leq S_f(\tau_2)$, следователно
\[
    s_f(\tau_1)\leq s_f(\tau^*) \leq S_f(\tau^*) \leq S_f(\tau_2)\square
\]

\[
    [s_f(\tau_1);S_f(\tau_1)]\cap [s_f(\tau_2);S_f(\tau_2)]=\varnothing, \forall\tau_1,\tau_2 \text{ - подразбивания на }[a,b]
\]

$f:[a,b]\rightarrow\mathbb{R}$, ограничена

$\upint\limits_{a}^{b} f := inf\{S_f(\tau): \tau - \text{ подразбиване на }[a,b]\}$

$\lowint\limits_{a}^{b} f := sup\{s_f(\tau): \tau - \text{ подразбиване на }[a,b]\}$

Oт Лема 2 $s_f(\tau_1)\leq S_f(\tau_2), \forall \tau_1,\tau_2$ подразбивания на $[a,b]$. Следователно
$\lowint\limits_{a}^{b} f \leq S_f(\tau_2), \forall \tau_2$ подразбиване на $[a,b]$ и $\lowint\limits_{a}^{b} f \leq \upint\limits_{a}^{b} f \leq$.

\opredelenie{Интегруемост по Риман}{
    $f: [a,b] \to \mathbb{R}$ се нарича нтегруема по Риман ако е ограничена и 
    $\lowint\limits_{a}^{b}f = \upint\limits_{a}^{b}f$. В този случай
    числото $\lowint\limits_{a}^{b}f = \upint\limits_{a}^{b}f$ се нарича
    Риманов интеграл по $f$ в $[a,b]$ и се бележи
    $\int_{a}^{b}f$ или $\int_{a}^{b}f(x)\,dx$
}

\primer{Функция на Дирихле}{}

\section*{Критерий за Интегруемост}
Твърдим, че: \\
\begin{center}
    $f$ е интегрируема по Риман \\ 
    $\xLeftrightarrow[]{\text{1}}$  \\
    $\forall \epsilon >0 \exists \tau_1, \tau_2$ - подр. на $[a,b]: Sf(\tau_1)-sf(\tau_2) < \epsilon$ \\
    $\xLeftrightarrow[]{\text{2}}$ \\
    $\forall \epsilon >0 \exists \tau$ - подр. на $[a,b]: Sf(\tau)-sf(\tau) < \epsilon$ \\
    $\iff$ \\
    $\forall \epsilon >0 \exists \tau$ - подр. на $[a,b]: \sum_{i=0}^{n}w(f,[x_{i-1}, x])$ \\
\end{center}

$(\xRightarrow[]{\text{1}})$ Фиксираме $\epsilon > 0$ \\
$\int_{a}^{b} + \frac{\epsilon}{2} = \upint\limits_{a}^{b}f + \frac{\epsilon}{2} > \upint\limits_{a}^{b}f \xRightarrow{} \exists \tau_1$ подр. на $[a,b], Sf(\tau_1)<\int_{a}^{b}f + \frac{\epsilon}{2}$ \\
$\int_{a}^{b} - \frac{\epsilon}{2} = \lowint\limits_{a}^{b}f - \frac{\epsilon}{2} < \lowint\limits_{a}^{b}f \xRightarrow{} \exists \tau_2$ подр. на $[a,b], sf(\tau_2)<\int_{a}^{b}f - \frac{\epsilon}{2}$ \\
$Sf(\tau_1)-sf(\tau_2) = \int_{a}^{b} + \frac{\epsilon}{2} - \int_{a}^{b} + \frac{\epsilon}{2}$ \\
$Sf(\tau_1)-sf(\tau_2) < \epsilon$

$(\xLeftarrow[]{\text{1}})$ Допускаме противното \\
$\forall \tau_1, \tau_2: Sf(\tau_1) - sf(\tau_2) \ge \upint\limits_{a}^{b}f - \lowint\limits_{a}^{b}f > 0$ \\

$(\xLeftarrow[]{\text{2}}) \tau_1 := \tau, \tau_2 := \tau$  \\

$(\xRightarrow[]{\text{2}})$ Фиксираме $\epsilon > 0$, $\tau_1,\tau_2 \xrightarrow{} Sf(\tau_1)-sf(\tau_2) < \epsilon$ \\
$\tau \ge \tau_1, \tau \ge \tau_2 \xRightarrow{} Sf(\tau)-sf(\tau) \le Sf(\tau_1) - sf(\tau_2) < \epsilon$

\end{document}